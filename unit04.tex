\documentclass[notes.tex]{subfiles}

\begin{document}
	\setcounter{chapter}{3}
	\chapter{Database Handling}
		\section{Creating a Database}
			\begin{definition}{Database}
				A collection of information that is organised so that it can be easily accessed, managed, and updated.

				Stores tables, indexes, foreign key constraints, primary key constraints, and other necessary components.

				A database table consists of rows and columns. Each column contains a single pieces of information, and a row is a collection of columns that contains complete information of an object, item, or entity.
			\end{definition}
			\begin{sidenote}{Syntax for Creating a Database}
				\verb|create database database_name;|
			\end{sidenote}
			\begin{sidenote}{MySQL Data Types}
				\begin{descriptimize}[nosep]
					\item[smallint, mediumint, int, bigint] Integer values
					\item[float] Single-precision floating-point values 
					\item[double] Double-precision floating-point values
					\item[char] Fixed-length strings up to 255 characters
					\item[varchar] Variable-length strings up to 255 characters
					\item[tinyblob, blob, mediumblob, longblob] Large blocks of binary data
					\item[tinytext, text, mediumtext, longtext] Long blocks of text data
					\item[date] Date values
					\item[time] Time values or durations
					\item[datetime] Combined date and time values    
				\end{descriptimize}
			\end{sidenote}
			\begin{sidenote}{Connecting to SQL server}
				\begin{minted}[autogobble]{python}
					import MySQLdb
					conn = MySQLdb.connect(host="localhost", user="user_name",
					                       passwd="password", db="db_name")
					cursor = conn.cursor()
				\end{minted}
			\end{sidenote}
			\begin{sidenote}{MySQLdb Methods}
				\begin{description}[nosep, style=nextline, leftmargin=1cm, font=\texttt]
					\item[MySQLdb.connect()] Connect to database server. (Returns database connection, normally variable is called \verb|conn|)Four parameters: host name, username, password, database name. Hostname specifies the location of the MySQL database server. For remote, specify the IP address. For local, use \verb|localhost|.
					\item[conn.cursor()] Returns the cursor object from the connection. Used to traverse records from the result set.
					\item[cursor.execute()] Execute an SQL query.
					\item[conn.commit()] Apply modifications to a database table.
					\item[conn.rollback()] Cancels all modifications applied to the database table. 
					\item[cursor.close()] Close cursor.
					\item[conn.close()] Disconnect database connection.
				\end{description}
			\end{sidenote}
			\begin{sidenote}{Some MySQL Commands}
				\begin{itemize}[nosep]
					\item \verb|use database_name;|
					\item \verb|show tables;|
					\item \verb|describe table_name;|
				\end{itemize}
			\end{sidenote}
			\begin{sidenote}{SQL INSERT}
				Used to insert rows into a table.
				\begin{minted}[autogobble]{sql}
					INSERT INTO table_name (table_column1, table_column2)
					VALUES (value1, value2);
				\end{minted}
			\end{sidenote}
			\pagebreak
			\subsection{Fetching Rows from a Table}
				Use the \verb|SELECT| SQL statement, via \verb|execute()| on a cursor object. A \verb|resultset| object is created that contains the tows from the database that satisfy the query.
				\begin{sidenote}{Result Set Cursor Methods}
					While a result set is created, you access it using the cursor. There are two possible methods:
					\begin{description}[nosep, font=\texttt]
						\item[cursor.fetchone()] Fetches the next row in result set.
						\item[cursor.fetchall()] Fetches the remaining rows in result set, or all, if none have been fetched.
					\end{description}
				\end{sidenote}
				You also use \verb|SELECT| to search for a specific value in the table.
				\begin{minted}[autogobble]{sql}
					SELECT * FROM table_name where attribute=value;
				\end{minted}
				You use UPDATE to change information, if it is found.
				\begin{minted}[autogobble]{sql}
					SELECT * from table_name where attribute=value;
					UPDATE table_name set attribute=value2 where attribute=value;
				\end{minted}
				You use DELETE to remove a row from a table.
				\begin{minted}[autogobble]{sql}
					SELECT * from table_name where attribute=value;
					DELETE from table_name where attribute=value;
				\end{minted}

		\section{Database Maintenance Through GUI Programs}
			To integrate and access databases in PyQt, you use the \verb|QSqlDatabase| class.
			\begin{sidenote}{QSqlDatabase Methods}
				\begin{descriptimize}[nosep]
					\item[\texttt{addDatabase()}] Specify the database driver of the database to which you want to establish a connection.
						\begin{itemize}[nosep]
							\item \textbf{QDB2}: IBM DB2
							\item \textbf{QMYSQL}: MySQL
							\item \textbf{QOCI}: Oracle Call Interface
							\item \textbf{QODBC}: ODBC (includes Microsoft SQL Server)
							\item \textbf{QPSQL}: PostgreSQL
							\item \textbf{QSQLITE}: SQLite version 3 or above
						\end{itemize}
					\item[\texttt{setHostName()}]
					\item[\texttt{setDatabaseName()}]
					\item[\texttt{setUserName()}]
					\item[\texttt{setPassword()}]
					\item[\texttt{open()}] Opens the database connection using the current connection attributes. Returns either \verb|True| or \verb|False|, depending on whether the connection to the database is successfully established or not.
					\item[\texttt{lastError()}] Display error information that may occur while opening a connection with the database through the \verb|open()| function.
				\end{descriptimize}
			\end{sidenote}

			\pagebreak
			\subsection{Displaying Rows}
				Use a \verb|QTableView| with a custom model.
				\begin{definition}{Model}
					A mirror image of the database table that the user can use to navigate and edit if required.
				\end{definition}
				The model used is an instance of the \verb|QSqlTableModel| class.
				\begin{definition}{QSqlTableModel}
					Provides a model that can be set to display information of a database table. Also makes it easy to navigate the model, and set editing strategy for the underlying database tables.
					\begin{sidenote}{Methods}
						\begin{description}[nosep, font=\texttt]
							\item[setTable()] Specify database table for the model.
							\item[setEditStrategy()] Applies the strategy for editing the database table.
								\begin{descriptimize}[nosep]
									\item[OnFieldChange] All modifications made in the model applied immediately to the database table.
									\item[OnRowChange] All modifications made to a row applied to the database table when moving to a different row.
									\item[OnManualSubmit] All modifications cached in the model and applied to the database table when \verb|submitAll()| is called. The modifications that have been cached can be cancelled or erased by called \verb|revertAll()|
								\end{descriptimize}
							\item[select()]
						\end{description}
					\end{sidenote}
				\end{definition}

		\ifSubfilesClassLoaded{%
			\vbox{\rulechapterend}}{\vspace*{\parskip}\rulebookend}
\end{document}